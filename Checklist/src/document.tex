\documentclass[letter, article]{article}
\usepackage{graphicx}
\usepackage{tcolorbox}
\usepackage{parskip}
\usepackage{lipsum}
\usepackage[hidelinks]{hyperref}
\usepackage[left=2cm,top=1.5cm,right=2cm,bottom=1.5cm]{geometry}

\begin{document}

\section{Practice}

\begin{itemize} \footnotesize
    \item Practice prior to our turn by swapping out the hopper for the forklift.  That is where we have been losing the most time during practice runs.  If you can do this switchout quickly you will finish all runs within the given time.
    \item Practice aligning the robot and launching a program quickly.  Do not drop the alignment pegs, sight down on the retracted pegs.
\end{itemize}

\section{Inspection}

\begin{enumerate} \footnotesize
    \item Hopper is loaded on the robot and the forklift attachment and robot are placed in the small inspection area.
\end{enumerate}

\section{Run 1}

\begin{enumerate}\footnotesize
    \item Turn off Bluetooth.
    \item Restart robot to calibrate gyro.
    \item Board is inspected for any incorrect setup.
        \begin{itemize} \footnotesize
            \item Do not touch the board.
            \item Tell the judge about anything that needs corrected.
            \item Be respectful to your team mates and the other team.  2--4 points are awarded for your behavior as a team for ``Gracious Professionalism''.
        \end{itemize}
    \item Hopper is attached.
    \item Side arm is attached and fully extended.
    \item Hopper arm is pointed straight out from front of robot.
    \item 3 grey bricks are loaded in the hopper.
    \item Robot is aligned in the starting position.
    \item Alignment pegs are flipped up.
    \item Run 1 code is selected for run.
    \item Start code on ``1, 2, 3, LEGO!''
\end{enumerate}

\begin{tcolorbox}[title=Note:]
\small If the robot does something wrong, pick it up quickly and either restart this run or start on the next run. 
\end{tcolorbox}

\section{Run 2}

\begin{enumerate}\footnotesize
    \item Hopper is removed.
    \item Forklift attachment is connected and in fully upright position.
    \item Robot is aligned in the starting position.
    \item Run 2 code is selected for run.
    \item Start the run.
\end{enumerate}

\begin{tcolorbox}[title=Note:]
\small The blue box will be pushed off the mat, set it aside and get the next run started before loading up the blue box.
\end{tcolorbox}

\section{Run 3}

\begin{enumerate}\footnotesize
    \item No attachment changes.
    \item Arm is in raised position.
    \item Robot is aligned in the starting position.
    \item Run 3 code is selected for run.
    \item Start the run.
\end{enumerate}

\end{document}
