\documentclass[letter, article]{article}
\usepackage{graphicx}
\usepackage{tcolorbox}
\usepackage{parskip}
\usepackage{lipsum}
\usepackage[hidelinks]{hyperref}
\usepackage[left=2cm,top=1.5cm,right=2cm,bottom=1.5cm]{geometry}

\begin{document}

\section{Run 1}

\begin{enumerate}
    \item Board is inspected for any incorrect setup.
        \begin{itemize}
            \item Do not touch the board.
            \item Tell the judge about anything that needs corrected.
        \end{itemize}
    \item Hopper is attached.
    \item Side arm is attached and fully extended.
    \item Hopper arm is pointed straight out.
    \item 3 grey bricks are loaded in the hopper.
    \item Robot is aligned in the starting position.
    \item Alignment pegs are flipped up.
    \item Run 1 code is selected for run.
    \item Start code on ``1, 2, 3, LEGO!''
\end{enumerate}

\begin{tcolorbox}[title=Note:]
If the robot does something wrong, pick it up quickly and either restart this run or start on the next run.  The first pickup has no point penalty!
\end{tcolorbox}

\section{Run 2}

\begin{enumerate}
    \item Hopper is removed.
    \item Forklift attachment is connected and in fully upright position.
    \item Robot is aligned in the starting position.
    \item Run 2 code is selected for run.
    \item Start the run.
\end{enumerate}

\begin{tcolorbox}[title=Note:]
The blue box will be pushed off the mat, set it aside and get the next run started before loading up the blue box.
\end{tcolorbox}

\section{Run 3}

\begin{enumerate}
    \item No attachment changes.
    \item Arm is in raised position.
    \item Robot is aligned in the starting position.
    \item Run 3 code is selected for run.
    \item Start the run.
\end{enumerate}

\begin{tcolorbox}[title=Note:]
Do a happy dance!
\end{tcolorbox}

\end{document}
